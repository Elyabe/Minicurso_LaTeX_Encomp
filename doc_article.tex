\documentclass[10pt,a4paper,twocolumn]{book}
%PREAMBULO
	
	\usepackage[utf8]{inputenc} % Acentuação 
	\usepackage{multirow} %Mesclagem vertical de linhas
	\usepackage{graphicx} %Inserção de figuras
	\usepackage{subfigure}	
	\usepackage[portuguese]{babel} %Tradução dos labels
	
\begin{document} 
%CORPO DO DOCUMENTO
	
	\tableofcontents
	
	\chapter{Minicurso de \LaTeX}
	
	Hello World! \\ 
	Introdução!
	Problemas na acentua\c{c}\~ao. Assim que se resolve no modo hard!
	
	\section{Fontes}
		\begin{enumerate}
			\item Tamanho
				\begin{itemize}
					\item {\tiny Pequena} 
					\item {\small Texto } 
					\item {\Large Oláaaa}  
					\item {\LARGE Big }
					\item {\Huge Oahdsad}
				\end{itemize}
			\item Estilo
				\begin{itemize}
					\item \textbf{Negrito} ou {\bf negrito mais negrito}
					\item \textit{Itálico} ou {\it Itálico mais mais}
					\item \underline{Sublinhado}
				\end{itemize}
			
		\end{enumerate}
	
	
	
	\section{Titulo da seção}
		Esta é minha primeira seção.
		\subsection{Subseção}
			Uma coleção de itens:
			\begin{itemize}
				\item Item 1
				\item Item 2
			\end{itemize}
			Quero uma enumeração
		
			\begin{enumerate}
				\item Bom
				\begin{enumerate}
					\item Um subitem
					\begin{itemize}
						\item Um subsubitem
					\end{itemize}
				\end{enumerate}
				\item Olá
				\item[labelPerson] Último item 
			\end{enumerate}
			
			\subsubsection{Subsubseção}
 	\section{Segunda Seção}
		Vamos trocar uma seção pela outra?
	
	\section{Outros elementos}
   		\begin{table}[h]
			\caption{Minha primeira tabela}
			\centering
			\begin{tabular}{ c | l | r } \hline
				\multicolumn{3}{c}{Texto mesclado} \\ \hline
				Centralizado &esquerda &direita \\ \hline
				linha 2 & & \\ 
				linha3 & & \\ \hline
			\end{tabular}
		\end{table}
		
		\begin{table}[h]
			\caption{Tabela com mesclagem de linhas}
			\centering
			\begin{tabular}{ c | l | r } \hline
				\multicolumn{3}{c}{Texto mesclado} \\ \hline
				Centralizado &esquerda &direita \\ \hline
				linha 2 & & \\ 
				linha3 & & \\ \hline
				\multirow{2}{*}{Celulas mescladas} & & \\ \hline
				& & \\
				& & \\ \hline
				linha 6 & &\\ \cline{2-3}
				linha 7 & &\\ \hline
			\end{tabular}
		\end{table}
		
	\section*{Vamos inserir uma imagem?}
	
		\begin{figure}[h]
			\caption{Legenda huehue}
			\includegraphics[scale=0.2]{figuras/encomp-logo.png}
		\end{figure}
	
		\begin{figure}[h]
			\caption{Subfiguras}
			\label{fig:subfiguras}
			\subfigure[Figura a]{\includegraphics[scale=0.1]{figuras/encomp-logo.png} \label{subfig:logo}}
			\subfigure[Figura  b ]{\includegraphics[scale=0.18]{figuras/joao-crazy.jpg}}
		\end{figure}
	
		A figura \ref{fig:subfiguras} mostra duas fig.
%	\part{Uma parte}
	
		\section{MathLovers} 
			Uma seção para morrer de amores.
			Que a $f = m \cdot a $ esteja com vocês.
			Na forma vetorial, fica $$\vec{f} = m \cdot \vec{a}.$$
			
			Num triângulo qualquer, vale a Lei dos Cossenos
			\begin{equation}
				a^2 + b^2 - 2\cos(\theta) = c^2\mbox{, com } \theta = \hat{A}.
				\label{eq:leiCos}
			\end{equation}
			Em particular, num triângulo retângulo, $\theta = \frac{\pi}{2}$.
			Daí
			\begin{math}
				 a^2 + b^2 = c^2
			\end{math}
			
			
		
\end{document}